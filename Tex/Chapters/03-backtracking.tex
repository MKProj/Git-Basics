\chapter{Backtracking in Git}
\section{Showing Latest Commit Logs}

In Git, the commit you are currently on is known as the \verb!HEAD! commit. The \verb!git show HEAD! command 
will output everything the \verb!git log! command display for the \verb!HEAD! commit and all the file changes that were made when committed. 

\section{Reset Using SHA}

Let's say you want to go back to a previous commit, how does one do that? Well you use the
\verb!git reset commit_SHA! command, and what it does is set the \verb!HEAD! commit to the \verb!commit_SHA! commit. 

To use \verb!git reset!, you will need to use the first seven digits of the previous commit's SHA, and 
that can be found using the \verb!git log! command to see records of previous commits.

\section{Removing A File from Staging}

Let's say you've added a file's changes to the staging area, but you change your mind, you can do the following, 
\verb!git reset HEAD <filename>!, and what it does is remove the file from the staging area.
\\
\textit{Note: This does not discard changes made in the working directory}
\\
You can use \verb!git status! to make sure your file was properly removed from the staging area. 

\section{Going Back to the Last Commit}
Let's say you want to go back to the changes made in a file in your last commit, or the \verb!HEAD! commit. 
To do this, you can use \verb!git checkout HEAD <filename>! and it will make the file in the working directory to go back to the changes made in the last commit.

You can make sure it worked by using \verb!git diff! command to see if the rollback was successful. If nothing is outputted, that means it worked. 
\newpage