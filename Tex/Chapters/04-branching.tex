\chapter{Branching in Git}
\section{The Master Branch}

When you initialize a git project, all work by default is done on the \verb!master! branch. When you make your first commit, the \verb!master! branch is automatically created.
\\\\
\textit{Note: You can have the master branch be main using} \verb!git branch -M main! \textit{after the first commit.}
\\\\
You can create new branches from the \verb!master! branch when you develop new features in your project or when you do testing. 
You can also see which branch you're on by using the \verb!git branch! command. 

\begin{verbatim}
# Here's an example of the Phaktionz-CLI branches 
    $ git branch 
    * beta
      edge
      main
      stable
# As can be seen, the master branch is switched to main 
# and the current branch being worked on is the beta branch.     
\end{verbatim}

\section{Creating a New Branch}
In Git, the \verb!git branch branch_name! command is used to create a new branch called \verb!branch_name!. 
Branches should be named something that describes the purpose of the branch.
\\
\textit{For example: The }\verb!beta!\textit{ branch represents the beta channel of Phaktionz-CLI.}

Also a branch name cannot contain white spaces: \verb!some_name! or \verb!some-name! are valid, however \verb!some name! is invalid. 

\section{Deleting a Branch}
In Git, you can delete a branch by using \verb!git branch -d branch_name!, and then the branch named, \verb!branch_name! will be removed.
\\
\textit{Note: Usually it's a good idea to merge the branch with the }\verb!master!\textit{ branch before deleting.}

\section{Merging a Branch}
In Git, if you would like to merge a branch with another, use the \verb!git merge! command. 
When you use the command \verb!git merge branch_name!, it will merge the branch, \verb!branch_name! to the branch you're currently on.

This is useful when your new feature works, so you can merge all the changes in that branch to your \verb!master! branch. 
\newpage