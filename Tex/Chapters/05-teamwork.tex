\chapter{Teamwork in Git}
\section{Git Remote}

When you work in a group, it is typical to collaborate with others using a Git server, or a website like Github (highly recommended).

A \verb!remote! is essentially a shared Git repository that allows for multiple collaborators.
Now keep in mind collaborators work independently, and merge their changes when ready. 

\section{Cloning a Remote Repository}
Now how do you get to work on your partner's git repository...on let's say your organization's github? Well why not clone the repository using \verb!git clone!.

It works locally:
\begin{verbatim}
$ ls
git-repository

# git clone remote-repo output-directory(optional)    
$ git clone git-project/ git-clone-project 
Cloning into 'git-clone-project'...
done. 
    
$ ls
git-clone-project git-project
\end{verbatim}
It works on remote sites, such as Github:
\begin{verbatim}
# HTTP Clone
$ git clone https://github.com/user/repo.git

# SSH Clone
$ git clone git@github.com:user/repo.git
\end{verbatim}

\section{Fetching Changes}
The command \verb!git fetch! downloads objects and changes from a remote repository \verb!origin!.

It however doesn't automatically merge these changes to your current branch, but instead keeps all the changes in a new branch \verb!origin/branch-name!.
\\
Now let's see this in action: 
\begin{verbatim}
$ git branch
* master

$ git fetch
remote: Counting objects: 3, done. 
remote: Compressing objects: 100% (3/3), done. 
remote: Total 3 (delta 1), reused 0 (delta 0)
Unpacking objects: 100% (3/3), done. 
From /home/user/git-project 
* [new branch]      master     ->  origin/master

$ git branch
* master
  remotes/origin/master
\end{verbatim}
To merge these changes, it's like last section, use the command, \verb!git merge! in the branch you want to have merge with the target branch.

To merge the fetch changes for our example, we do \verb!git merge origin/master! on branch \verb!master!. 

\section{Pushing Changes}
In Git, the \verb!git push origin branch-name! command pushes the branch \verb!branch-name!, and all of the committed changes, 
to the remote \verb!origin!. This branch can now be reviewed and fetched by collaborators.

\section{Looking at the Remotes}
In Git, the \verb!git-remote -v! command returns a list of remote repositories that the current project is connected to.
\begin{itemize}
    \item Git lists the name of the remote repository as well as its locations.
    \item Git automatically names this remote origin, because it refers to the remote repository of origin.
        \begin{itemize}
            \item However, it is possible to safely change its name.
        \end{itemize}
    \item The remote is listed twice: once for (fetch) and once for (push).
\end{itemize}

\begin{verbatim}
$ git-remote -v 
origin  /home/user/git-project/
(fetch)
origin  /home/user/git-project/
(push)
\end{verbatim}
        
    
\section{A Typical Collaboration Workflow}
A typical collaboration workflow is:
\begin{enumerate}
    \item Fetch and merge changes from the remote
    \item Create a branch to work on a new project feature
    \item Develop the feature on a branch and commit the work
    \item Fetch and merge from the remote again (in case new commits were made)
    \item Push branch up to the remote for review
\end{enumerate}
\textit{Note: Steps 1 and 4 are a safeguard against merge conflicts, which occur when two branches contain file changes that cannot be merged with the git merge command.}
\\\\
Here's a bash script that can be made to ease the need of the many commands: 

\begin{verbatim}
#!/bin/bash
git fetch
git merge origin/master 

echo "New Branch Name: "
read branch

git branch $branch
git checkout $branch

git fetch
git add * 

echo "Message: "
read message

git commit -m $message

git merge origin/master

git push -u origin $branch
\end{verbatim}